\abstract{
    Najjednostavniji epiciklus predstavlja krug \v{c}iji se centar kre\'c{}e po kru\v{z}nici drugog kruga. Slo\v{z}eni epiciklusi nastaju rekurzivnim dodavanjem krugova u pomenuti sistem. Epiciklusi su poznati jo\v{s} od vremena starih Grka i kori\v{s}\'c{}eni su da opi\v{s}u slo\v{z}ena kretanja nebeskih tela. Pokazano je da se slaganjem dovoljnog broja epiciklusa odgovaraju\'c{}ih dimenzija i njihovim kretanjem odgovaraju\'c{}im brzinama mogu iscrtati najrazli\v{c}itije orbite bez obzira na njihovu slo\v{z}enost. U ovom radu se opisuje veza izmedju epiciklusa i diskretne Furijeove transformacije i ista se koristi za crtanje proizvoljnih neprekidnih linija daju\'c{}i fascinantno geometrijsko shvatanje Furijeove transformacije koje se krilo u epiciklusima poznatim od davnina.
}